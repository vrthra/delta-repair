\documentclass[sigconf,review,anonymous]{acmart}


\usepackage{xspace}
\usepackage{array,multirow,graphicx}
\usepackage{float}
\usepackage{framed}
\setlength{\FrameSep}{3pt}
\setlength{\OuterFrameSep}{2pt}
\newenvironment{result}{\begin{framed}\centering\it}{\end{framed}}


\newcommand{\todo}[1]{\textcolor{red}{\textbf{TODO: }\emph{#1}}}
\newcommand{\com}[1]{\textcolor{orange}{\textbf{COMMENT: }\emph{#1}}}
\newcommand{\recheck}[1]{\textcolor{red}{#1}}
\newcommand{\revise}[1]{\textcolor{blue}{#1}}


\newcommand{\approach}{\textsc{Approach}\xspace}
\def\bfr{bFuzzerRepairer\xspace}
\def\ddmin{DDMin\xspace}

%%
%% \BibTeX command to typeset BibTeX logo in the docs
\AtBeginDocument{%
  \providecommand\BibTeX{{%
    \normalfont B\kern-0.5em{\scshape i\kern-0.25em b}\kern-0.8em\TeX}}}

%% Rights management information.  This information is sent to you
%% when you complete the rights form.  These commands have SAMPLE
%% values in them; it is your responsibility as an author to replace
%% the commands and values with those provided to you when you
%% complete the rights form.
\setcopyright{acmcopyright}
\copyrightyear{2018}
\acmYear{2018}
\acmDOI{XXXXXXX.XXXXXXX}

%% These commands are for a PROCEEDINGS abstract or paper.
\acmConference[ESEC/FSE 2022]{The 30th ACM Joint European Software Engineering Conference and Symposium on the Foundations of Software Engineering}{14 - 18 November, 2022}{Singapore}
\acmPrice{15.00}
\acmISBN{978-1-4503-XXXX-X/18/06}

\usepackage[utf8]{inputenc}

\begin{document}

\title{Input Debugging via Rich %and Fast 
Failure Feedback}


\author{Ben Trovato}
\authornote{Both authors contributed equally to this research.}
\email{trovato@corporation.com}
\orcid{1234-5678-9012}
\author{G.K.M. Tobin}
\authornotemark[1]
\email{webmaster@marysville-ohio.com}
\affiliation{%
  \institution{Institute for Clarity in Documentation}
  \streetaddress{P.O. Box 1212}
  \city{Dublin}
  \state{Ohio}
  \country{USA}
  \postcode{43017-6221}
}

\author{Lars Th{\o}rv{\"a}ld}
\affiliation{%
  \institution{The Th{\o}rv{\"a}ld Group}
  \streetaddress{1 Th{\o}rv{\"a}ld Circle}
  \city{Hekla}
  \country{Iceland}}
\email{larst@affiliation.org}

\author{Valerie B\'eranger}
\affiliation{%
  \institution{Inria Paris-Rocquencourt}
  \city{Rocquencourt}
  \country{France}
}

\author{Aparna Patel}
\affiliation{%
 \institution{Rajiv Gandhi University}
 \streetaddress{Rono-Hills}
 \city{Doimukh}
 \state{Arunachal Pradesh}
 \country{India}}

\author{Huifen Chan}
\affiliation{%
  \institution{Tsinghua University}
  \streetaddress{30 Shuangqing Rd}
  \city{Haidian Qu}
  \state{Beijing Shi}
  \country{China}}

\author{Charles Palmer}
\affiliation{%
  \institution{Palmer Research Laboratories}
  \streetaddress{8600 Datapoint Drive}
  \city{San Antonio}
  \state{Texas}
  \country{USA}
  \postcode{78229}}
\email{cpalmer@prl.com}

\author{John Smith}
\affiliation{%
  \institution{The Th{\o}rv{\"a}ld Group}
  \streetaddress{1 Th{\o}rv{\"a}ld Circle}
  \city{Hekla}
  \country{Iceland}}
\email{jsmith@affiliation.org}

\author{Julius P. Kumquat}
\affiliation{%
  \institution{The Kumquat Consortium}
  \city{New York}
  \country{USA}}
\email{jpkumquat@consortium.net}

%%
%% By default, the full list of authors will be used in the page
%% headers. Often, this list is too long, and will overlap
%% other information printed in the page headers. This command allows
%% the author to define a more concise list
%% of authors' names for this purpose.
\renewcommand{\shortauthors}{Trovato and Tobin, et al.}

\date{February 2022}

\begin{abstract}
%Problem/
%\textbf{Context:} 
%\revise{
Program failures are sometimes induced by %when 
\textit{faulty inputs}, rather than \textit{buggy programs}, % are fed to \textit{valid} programs, 
e.g., 
%This can be 
due to incomplete or corrupted data. When an input \textit{solely} %is responsible for the program 
induces a failure in a valid program, the developer is saddled with the task of 
\textit{input debugging}.  %}
%\textbf{Objective:}  
%\revise{
Debugging \textit{failure-inducing inputs} involves identifying the  \textit{root cause} of the failure (e.g., the faulty input fragment)
% causing the failure), 
and \textit{repairing the input} such that it can be processed by the program. This is particularly difficult for structured inputs with complex input specifications (e.g., JSON).  
%} 
%\textbf{Methodology:} 
%\revise{
In this work, we present a \textit{language-agnostic, black-box} testing approach (called \approach) that leverages 
%ich and fast 
\textit{failure feedback} %from the program 
to automatically debug faulty inputs. The key insight of our approach %(called \approach) 
is to \textit{semantically} repair faulty inputs 
%to debug inputs via %by employing %a 
%combination of 
%\textit{test experimentation} that leverages 
using %leverage 
the richness %and speed of 
\textit{failure feedback}, %to \textit{semantically}, 
such as 
%Specifically, \approach uses program feedback to 
%%. \approach employs  to identify %ing input properties such as the 
%identify 
the \textit{validity, incompleteness and incorrectness} checks of input fragments. 
Given a failure-inducing input and a valid program,  
\approach conducts test experiments semantic checks to identify %ies 
faulty input fragments. 
%by conducting semantic checks of input fragments (e.g., (in)completeness) using test experiments. 
It then repairs the faulty input by removing or \textit{synthesizing} candidate input elements. % for repair.  
\revise{In our evaluation, \todo{\approach repaired ... recovered  ...} 
In addition, \approach overcomes the span length issue of (lexical) DDmax and outperforms DDMax by X\%, %despite %   %} In addition, \approach addresses several limitations of the state of the art, e.g., it overcomes the span length issue of (lexicial) DDmax and it does 
without access to an input grammar (syntactic DDmax). }

%\textbf{Conclusion:}


\end{abstract}

%%
%% The code below is generated by the tool at http://dl.acm.org/ccs.cfm.
%% Please copy and paste the code instead of the example below.
%%
% \begin{CCSXML}
% <ccs2012>
%  <concept>
%   <concept_id>10010520.10010553.10010562</concept_id>
%   <concept_desc>Computer systems organization~Embedded systems</concept_desc>
%   <concept_significance>500</concept_significance>
%  </concept>
%  <concept>
%   <concept_id>10010520.10010575.10010755</concept_id>
%   <concept_desc>Computer systems organization~Redundancy</concept_desc>
%   <concept_significance>300</concept_significance>
%  </concept>
%  <concept>
%   <concept_id>10010520.10010553.10010554</concept_id>
%   <concept_desc>Computer systems organization~Robotics</concept_desc>
%   <concept_significance>100</concept_significance>
%  </concept>
%  <concept>
%   <concept_id>10003033.10003083.10003095</concept_id>
%   <concept_desc>Networks~Network reliability</concept_desc>
%   <concept_significance>100</concept_significance>
%  </concept>
% </ccs2012>
% \end{CCSXML}

% \ccsdesc[500]{Computer systems organization~Embedded systems}
% \ccsdesc[300]{Computer systems organization~Redundancy}
% \ccsdesc{Computer systems organization~Robotics}
% \ccsdesc[100]{Networks~Network reliability}

%%
%% Keywords. The author(s) should pick words that accurately describe
%% the work being presented. Separate the keywords with commas.
\keywords{input debugging, input repair, program failures, structured inputs}


%% A "teaser" image appears between the author and affiliation
%% information and the body of the document, and typically spans the
%% page.
% \begin{teaserfigure}
%   \includegraphics[width=\textwidth]{sampleteaser}
%   \caption{Seattle Mariners at Spring Training, 2010.}
%   \Description{Enjoying the baseball game from the third-base
%   seats. Ichiro Suzuki preparing to bat.}
%   \label{fig:teaser}
% \end{teaserfigure}

%%
%% This command processes the author and affiliation and title
%% information and builds the first part of the formatted document.
\maketitle

\section{Introduction}

\section{Overview}

\section{Approach}

\section{Experimental Setup}
\label{sec:experimental-setup}

\smallskip\noindent
\textbf{Research Questions: } 
\revise{
XXXXX 
}


\noindent
\textbf{RQ1 Effectiveness:} \revise{How effective is our approach (\approach) in repairing failure-inducing inputs?}


\noindent
\textbf{RQ2 Data Recovery and Loss:} \revise{
}

\noindent
\textbf{RQ3 Efficiency:} \revise{
}


\noindent
\textbf{RQ3 Diagnostic Quality:} \revise{
}


\noindent
\textbf{RQ4 Baseline Comparison (DDmax):} \revise{How does \approach compare to the ... 
}



\noindent
\textbf{Baseline (DDMax):}

\noindent
\textbf{Subject Programs and Commits:}


\noindent
\textbf{Metrics and Measure}

\noindent
\textbf{Implementation Details and Platform:}

\noindent
\textbf{Research Protocol:} 


\section{Experimental Results}

\section{Discussion}

\section{Threats to Validity}
Our approach (\approach) and empirical evaluations may be limited by the following validity threats:

\noindent
\textbf{External Validity:}

\noindent
\textbf{Internal Validity:}


\noindent
\textbf{Construct Validity:}


\section{Related Work}
\label{sec:related_work}
%TODO In this chapter, we discuss...

\begin{description}

\item[Input Rectification] is the process of transforming misbehaving inputs into inputs that behave predictably in the scope of a certain software system.
    In \textit{Automatic Input Rectification}~\cite{Long:2012:AIR:2337223.2337233} and \emph{Living in the Comfort Zone}~\cite{Rinard:2007:LCZ:1297027.1297072}, input constraints are learned from valid inputs to solve this problem, which are used to transform malicious inputs into rectified inputs that satisfy learned constraints.
    We do not employ such constraint learning techniques in \bfr.
    Instead, we employ the knowledge of a grammar describing the input file format and the feedback of a subject program to transform the input files into an acceptable subset.
    Instead of transforming inputs to comply to security-critical constraints, our goal is to recover as much of the input file as possible.

\item[File Recovery] aims to recover files that cannot be opened with a subject program due to corruption.
In \emph{S-DAGs}~\cite{scheffczyk2004s}, semantic consistency constraints are enforced on broken input files in a semi-automatic technique.
\emph{Docovery}~\cite{docovery:ase14} takes a similar approach using symbolic execution to change broken inputs to take error-free paths in the subject program.
While this is a whitebox approach that facilitates program analysis, \bfr relies on the feedback of the subject program in a black-box approach that results from executing the broken inputs.

\item[Input Debugging], in contrast to program debugging, aims to change inputs to localize faults in subject programs.
    There is numerous work that focuses on simplifying failure-inducing inputs~\cite{Zeller:2002:SIF:506201.506206, clause2009penumbra, hierarchicalDD, sterling2007automated}.
    Closely related to \bfr are particularly \cite{hierarchicalDD} and \cite{sterling2007automated}, which both aim to minimize the input files while still triggering a bug in the subject program, just like \ddmin.
    In contrast, \bfr tries to maximize the inputs while avoiding to trigger the bug in the subject program.

\item[Data Diversity] \cite{data_diversity} transforms a failure-inducing input into an input that can be processed by a subject program while generating the same output.
Their approach analyzes which regions of the input cause the fault and changes those regions to avoid the fault.
An important difference to \bfr is that we do not need any program analysis, but analyze the output of the subject program alone.

\item[Data Structure Repair] iteratively fixes corrupted data structures by enforcing they conform to consistency constraints~\cite{Demsky:2003:ADR:949343.949314, 1553560, hussain2010dynamic, Demsky:2006:IED:1146238.1146266}.
These constraints can be extracted, specified and enforced with predicates~\cite{elkarablieh2008juzi}, model-based systems~\cite{Demsky:2003:ADR:949343.949314}, goal-directed reasoning~\cite{1553560}, dynamic symbolic execution~\cite{hussain2010dynamic} or invariants~\cite{Demsky:2006:IED:1146238.1146266}.
Similarly to our approach, one goal in data structure repair is to successfully execute a subject program on broken input files safely and acceptably.
However, \bfr focuses on repairing inputs while recovering as much data as possible to avoid inducing faults in the subject program.

\item[Syntactic Error Recovery] is the implementation of error recovery schemes, typically in parsers and compilers~\cite{hammond1984survey, backhouse1979syntax}.
In most of the syntactic recovery approaches, various different techniques are employed, including insertion, deletion and replacement of symbols~\cite{anderson1981locally, cerecke2003locally, anderson1983assessment}, extending forward or backwards from a parser error~\cite{burke1982practical, mauney1982forward}, or more general methods of recovery and diagnosis~\cite{krawczyk1980error, aho1972minimum}.
Different to \bfr, these schemes ensure the compiler does not halt while parsing, while our approach aims to fix the input inducing the failure.

\item[Data Cleaning and Repair] is typically performed on complex database systems.
In most approaches, this includes analyzing the database to remove noisy data or fill in missing data~\cite{xiong2006enhancing, hernandez1995merge}.
In other approaches, developers are allowed to write and apply logical rules on the database~\cite{pochampally2014fusing, galhardas2000ajax, golab2010data, jeffery2006pipelined, raman2001potter, luebbers2003systematic}.
While all of these approaches repair database systems, \bfr repairs raw user inputs.

\item[Data Testing and Debugging] aims to identify program errors caused by well-formed but incorrect data while a user modifies a database~\cite{mucslu2013data}.
In continuous data testing (CDT)~\cite{mucslu2015preventing}, likely data errors are identified by continuously executing domain-specfic test queries, in order to warn users of test failures.
\emph{DATAXRAY}~\cite{wang2015error} also investigates the underlying conditions that cause data bugs, it reveals hidden connections and common properties among data errors.
In contrast to \bfr, these approaches aim to guard data from new errors by detecting data errors in database systems during modification.
\end{description}

\section{Conclusion}
\todo{XXXX }

\revise{We provide our tool, data and experimental results for easy replication, scrutiny and reuse:
}

 \begin{center}
 \vspace{-0.2mm}
     \textbf{\url{https://github.com/XXX}}
 \end{center}
 
% \begin{acks}
% % To Robert, for the bagels and explaining CMYK and color spaces.
% \end{acks}

%%
%% The next two lines define the bibliography style to be used, and
%% the bibliography file.
\bibliographystyle{ACM-Reference-Format}
\bibliography{bibliography}


\end{document}
